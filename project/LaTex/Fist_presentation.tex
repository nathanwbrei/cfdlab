\documentclass[10pt,a4paper]{beamer}
\usepackage[utf8]{inputenc}
\usepackage[english]{babel}
\usepackage{amsmath}
\usepackage{amsfonts}
\usepackage{amssymb}
\usepackage{graphicx}
\usepackage[breakable, theorems, skins]{tcolorbox}
\usepackage{multimedia}
\author{ Nathan Brei, Irving Cabrera, Natalia Saiapova}
\institute[TUM]{TU München}
\date{\today}
\title{Free surfaces with Lattice-Boltzmann method}
\titlegraphic{\includegraphics[height=4cm]{water-drop}}
\DeclareRobustCommand{\mybox}[2][gray!20]{%
\begin{tcolorbox}[   %% Adjust the following parameters at will.
        breakable,
        left=0pt,
        right=0pt,
        top=0pt,
        bottom=0pt,
        colback=#1,
        colframe=#1,
        width=\dimexpr\textwidth\relax, 
        enlarge left by=0mm,
        boxsep=5pt,
        arc=0pt,outer arc=0pt,
        ]
        #2
\end{tcolorbox}
}
\begin{document}
\begin{frame}
	\maketitle
\end{frame}
\begin{frame}
	\frametitle{Output from previous worksheets}
	\begin{itemize}
	\item LBM D3Q19;
	\item Different boundary conditions
		\begin{itemize}
		\item Non-Slip;  
		\item Inflow;
		\item Free-Slip.
		\end{itemize}
	\item 2D complex geometries reader.
	\end{itemize}
\end{frame}

\begin{frame}
  \begin{itemize}
    \frametitle{New values}
  \item New cell types: \textbf{GAS} and \textbf{INTERFACE};
  \item New values for cell: \textbf{mass} (double) and \textbf{fluid fraction} (double).\\
    Fluid fraction is
    \(
    \epsilon = \frac{m}{\rho},
    \)
    where $\rho$ is a density.
  \end{itemize}
\end{frame}
\begin{frame}
  \frametitle{Boundary conditions and gravity}
  \begin{itemize}
  \item Add gravity to our model;
  \item Reconstruct distributions from GAS cells to INTERFACE.\\
    Use INFLOW boundary conditions.
  \end{itemize}
\end{frame}

\begin{frame}
  \frametitle{Mass exchange}
  Calculate new mass for INTERFACE cells.
    \begin{equation}
      m\big(x, t+\bigtriangleup t\big) = m\big(x,t\big)+\sum_{i=1}\bigtriangleup m_i
    \end{equation}
    where $\bigtriangleup m_i$ is calculated only between INTERFACE-INTERFACE or
    FLUID-INTERFACE cells in the following way:
    \begin{equation}
      \bigtriangleup m_i = \Big(f_{\bar{i}}\big(x+\bigtriangleup te_i,t\big)-f_i\big(x,t\big)\Big)\frac{\epsilon\big(x+\bigtriangleup te_i,t\big)+\epsilon\big(x,t\big)}{2}
    \end{equation}
    For fluid cell mass is equal to its density.
\end{frame}

\begin{frame}
  \frametitle{Flag field update}
  Update fraction field;
  \mybox[green!20]{
    \textbf{Important rule}\\
    Between GAS and FLUID cells \textbf{always} has to be
    an INTERFACE cell.
  }
  Update flag field afterwards. 
\end{frame}
\begin{frame}
  \frametitle{What can be done}
  \begin{itemize}
  \item Make beautiful examples with higher resolution (falling drop, obstacles
    etc.);
    \item Improve physical correctness;
  \item Make our code faster.
  \end{itemize}
\end{frame}

\begin{frame}
\frametitle{Thank you for your attention!}
\movie [width=7cm,height=5cm,duration=5s,poster]{}{wall100-freeslip.mp4} 
\end{frame}


\begin{frame}
  References
  \begin{itemize}
  \item Physically based Animation of Free Surface Flows with the Lattice
    Boltzmann Method, Nils Thürey, 2007.
  \item GPU based simulation and visualization of fluids with free surfaces, Martin Schreiber, 2010.
  \end{itemize}
\end{frame}


\end{document}
