\documentclass{article}
\usepackage{graphicx}

\begin{document}

\title{Free Surface with LBM D3Q19}
\author{Group 6}

\maketitle

\section{LALALL}\label{free-surfaces-with-lbm-d3q19}

Project for Computational Fluid Dynamics Lab course.

Group 6: Nathan Brei, Irving Cabrera, Natalia Saiapova.

\subsection{Compilation.}\label{compilation.}

Requirements: UNIX based OS.

Use \texttt{make} command to compile:

\begin{verbatim}
>> make
\end{verbatim}

There is debug mode available. This mode includes additional checks for
velocity and density, computes total mass in the domain and checks that
there is no FLUID cell that is neighboring with a GAS cell directly.
Also it calculates a time spent on the following parts: streaming,
collision, flag updating and boundary treatment.

\begin{verbatim}
>> make debug
\end{verbatim}

\subsection{Usage}\label{usage}

To run the program use

\begin{verbatim}
>> ./lbsim <exampleName> <number of threads>
\end{verbatim}

For each example it is required to have
\texttt{\textless{}exampleName\textgreater{}.dat} and
\texttt{\textless{}exampleName\textgreater{}.pgm} files. They have to be
placed in the \texttt{examples} directory.\\
\texttt{\textless{}number\ of\ threads\textgreater{}} specifies how many
threads will be used by OpenMP.

Output is vtk-files and they can be found in the \texttt{vtk-output}
directory (if thare is no such directory it will be created).

\subsubsection{Description of input
files}\label{description-of-input-files}

We added a few new fields to the \texttt{input.dat} file:

\toprule
\begin{minipage}[b]{0.05\columnwidth}\raggedright\strut
Parameter\strut
\end{minipage} & \begin{minipage}[b]{0.05\columnwidth}\raggedright\strut
Description\strut
\end{minipage}\tabularnewline
\midrule
\endhead
\begin{minipage}[t]{0.05\columnwidth}\raggedright\strut
radius\strut
\end{minipage} & \begin{minipage}[t]{0.05\columnwidth}\raggedright\strut
Radius of a droplet\strut
\end{minipage}\tabularnewline
\begin{minipage}[t]{0.05\columnwidth}\raggedright\strut
exchange\_factor\strut
\end{minipage} & \begin{minipage}[t]{0.05\columnwidth}\raggedright\strut
Factor to increase speed of a mass exchnge (mainly to get more beautiful
videos)\strut
\end{minipage}\tabularnewline
\begin{minipage}[t]{0.05\columnwidth}\raggedright\strut
forces\_x\strut
\end{minipage} & \begin{minipage}[t]{0.05\columnwidth}\raggedright\strut
Gravity force in the \texttt{x} direction\strut
\end{minipage}\tabularnewline
\begin{minipage}[t]{0.05\columnwidth}\raggedright\strut
forces\_y\strut
\end{minipage} & \begin{minipage}[t]{0.05\columnwidth}\raggedright\strut
Gravity force in the \texttt{y} direction\strut
\end{minipage}\tabularnewline
\begin{minipage}[t]{0.05\columnwidth}\raggedright\strut
forces\_z\strut
\end{minipage} & \begin{minipage}[t]{0.05\columnwidth}\raggedright\strut
Gravity force in the \texttt{z} direction\strut
\end{minipage}\tabularnewline
\bottomrule
\end{longtable}

\subsection{What we have done}\label{what-we-have-done}

\paragraph{New cell types}\label{new-cell-types}

First of all we added 2 new cell types: \texttt{FLUID} and
\texttt{INTERFACE}.\\
Here is table representing cell types, which should be used in
\texttt{\textless{}exampleName\textgreater{}.pgm} file:

\begin{longtable}[]{@{}llll@{}}
\toprule
Cell type & Value & Cell type & Value\tabularnewline
\midrule
\endhead
FLUID & 0 & PRESSURE\_IN & 5\tabularnewline
INTERFACE & 1 & OBSTACLE & 6\tabularnewline
MOVING\_WALL & 2 & FREESLIP & 7\tabularnewline
INFLOW & 3 & NOSLIP & 8\tabularnewline
OUTFLOW & 4 & GAS & 9\tabularnewline
\bottomrule
\end{longtable}

\paragraph{Test functions}\label{test-functions}

In the \texttt{checks.c} one can find three check functions:

\begin{enumerate}
\tightlist
\item
  \texttt{check\_in\_rank} checks that every \texttt{FLUID} cell has a
  density between \texttt{0.9} and \texttt{1.1} and norm of the velocity
  vector is less then \texttt{sqrt(3)\ *\ C\_S};
\item
  \texttt{check\_flags} checks that no \texttt{FLUID} cell has a
  \texttt{GAS} neighbor;
\item
  \texttt{check\_mass} compute total mass in the domain (\texttt{FLUID}
  and \texttt{INTERFACE} cells).
\end{enumerate}

This checks will be activated in debug mode and be executed every time
when we write output file.

\paragraph{External forces influence}\label{external-forces-influence}

In collision step additional term for external forces was added.\\
Related function in \texttt{collision.c}:
\texttt{computeExternal(int\ i,\ float\ density,\ float\ *\ extForces)},
where \texttt{extForces} is a force vector specified in input file.

\paragraph{Mass and fluid fraction
fields}\label{mass-and-fluid-fraction-fields}

For every cell we store 2 additional float values: mass and fluid
fraction. We update mass field during streaming step. Fluid fraction
field is updated during collide step.\\
To be able get more fast we introduced \texttt{exchange\_factor} in an
input file. By this factor we increase mass exchange between cells.

\paragraph{DF from gas cells}\label{df-from-gas-cells}

In the streaming step was added reconstruction from \texttt{GAS} cells.
Streaming works normally for all \texttt{FLUID} cells amd for
\texttt{INTERFACE} cells we are checking whether a neighbor is GAS. If
yes, we are using outflow boundary formula in this direction.

\paragraph{Flag update}\label{flag-update}

\paragraph{Parallelization with
OpenMP}\label{parallelization-with-openmp}

We have added \texttt{pragma\ omp\ parallel\ for} to streaming,
collision, treatment boundaries and update flags.

\paragraph{Other optimizations}\label{other-optimizations}

At the beginning our main bottleneck was \texttt{computeFeq} function
which took approximately 42\% percent of the whole time. We splitted the
Q-loop in this function in a way that compiler was able to vectorize it,
then we got rid of all divisions and replaced them by corresponding
multiplications. Then we changed types of all double fields to float. It
gave us aproximately 2x speedup.
\section{Free Surfaces with LBM
D3Q19}\label{free-surfaces-with-lbm-d3q19}

Project for Computational Fluid Dynamics Lab course.

Group 6: Nathan Brei, Irving Cabrera, Natalia Saiapova.

\subsection{Compilation.}\label{compilation.}

Requirements: UNIX based OS.

Use \texttt{make} command to compile:

\begin{verbatim}
>> make
\end{verbatim}

There is debug mode available. This mode includes additional checks for
velocity and density, computes total mass in the domain and checks that
there is no FLUID cell that is neighboring with a GAS cell directly.
Also it calculates a time spent on the following parts: streaming,
collision, flag updating and boundary treatment.

\begin{verbatim}
>> make debug
\end{verbatim}

\subsection{Usage}\label{usage}

To run the program use

\begin{verbatim}
>> ./lbsim <exampleName> <number of threads>
\end{verbatim}

For each example it is required to have
\texttt{\textless{}exampleName\textgreater{}.dat} and
\texttt{\textless{}exampleName\textgreater{}.pgm} files. They have to be
placed in the \texttt{examples} directory.\\
\texttt{\textless{}number\ of\ threads\textgreater{}} specifies how many
threads will be used by OpenMP.

Output is vtk-files and they can be found in the \texttt{vtk-output}
directory (if thare is no such directory it will be created).

\subsubsection{Description of input
files}\label{description-of-input-files}

We added a few new fields to the \texttt{input.dat} file:

\begin{longtable}[]{@{}ll@{}}
\toprule
\begin{minipage}[b]{0.05\columnwidth}\raggedright\strut
Parameter\strut
\end{minipage} & \begin{minipage}[b]{0.05\columnwidth}\raggedright\strut
Description\strut
\end{minipage}\tabularnewline
\midrule
\endhead
\begin{minipage}[t]{0.05\columnwidth}\raggedright\strut
radius\strut
\end{minipage} & \begin{minipage}[t]{0.05\columnwidth}\raggedright\strut
Radius of a droplet\strut
\end{minipage}\tabularnewline
\begin{minipage}[t]{0.05\columnwidth}\raggedright\strut
exchange\_factor\strut
\end{minipage} & \begin{minipage}[t]{0.05\columnwidth}\raggedright\strut
Factor to increase speed of a mass exchnge (mainly to get more beautiful
videos)\strut
\end{minipage}\tabularnewline
\begin{minipage}[t]{0.05\columnwidth}\raggedright\strut
forces\_x\strut
\end{minipage} & \begin{minipage}[t]{0.05\columnwidth}\raggedright\strut
Gravity force in the \texttt{x} direction\strut
\end{minipage}\tabularnewline
\begin{minipage}[t]{0.05\columnwidth}\raggedright\strut
forces\_y\strut
\end{minipage} & \begin{minipage}[t]{0.05\columnwidth}\raggedright\strut
Gravity force in the \texttt{y} direction\strut
\end{minipage}\tabularnewline
\begin{minipage}[t]{0.05\columnwidth}\raggedright\strut
forces\_z\strut
\end{minipage} & \begin{minipage}[t]{0.05\columnwidth}\raggedright\strut
Gravity force in the \texttt{z} direction\strut
\end{minipage}\tabularnewline
\bottomrule
\end{longtable}

\subsection{What we have done}\label{what-we-have-done}

\paragraph{New cell types}\label{new-cell-types}

First of all we added 2 new cell types: \texttt{FLUID} and
\texttt{INTERFACE}.\\
Here is table representing cell types, which should be used in
\texttt{\textless{}exampleName\textgreater{}.pgm} file:

\begin{longtable}[]{@{}llll@{}}
\toprule
Cell type & Value & Cell type & Value\tabularnewline
\midrule
\endhead
FLUID & 0 & PRESSURE\_IN & 5\tabularnewline
INTERFACE & 1 & OBSTACLE & 6\tabularnewline
MOVING\_WALL & 2 & FREESLIP & 7\tabularnewline
INFLOW & 3 & NOSLIP & 8\tabularnewline
OUTFLOW & 4 & GAS & 9\tabularnewline
\bottomrule
\end{longtable}

\paragraph{Test functions}\label{test-functions}

In the \texttt{checks.c} one can find three check functions:

\begin{enumerate}
\tightlist
\item
  \texttt{check\_in\_rank} checks that every \texttt{FLUID} cell has a
  density between \texttt{0.9} and \texttt{1.1} and norm of the velocity
  vector is less then \texttt{sqrt(3)\ *\ C\_S};
\item
  \texttt{check\_flags} checks that no \texttt{FLUID} cell has a
  \texttt{GAS} neighbor;
\item
  \texttt{check\_mass} compute total mass in the domain (\texttt{FLUID}
  and \texttt{INTERFACE} cells).
\end{enumerate}

This checks will be activated in debug mode and be executed every time
when we write output file.

\paragraph{External forces influence}\label{external-forces-influence}

In collision step additional term for external forces was added.\\
Related function in \texttt{collision.c}:
\texttt{computeExternal(int\ i,\ float\ density,\ float\ *\ extForces)},
where \texttt{extForces} is a force vector specified in input file.

\paragraph{Mass and fluid fraction
fields}\label{mass-and-fluid-fraction-fields}

For every cell we store 2 additional float values: mass and fluid
fraction. We update mass field during streaming step. Fluid fraction
field is updated during collide step.\\
To be able get more fast we introduced \texttt{exchange\_factor} in an
input file. By this factor we increase mass exchange between cells.

\paragraph{DF from gas cells}\label{df-from-gas-cells}

In the streaming step was added reconstruction from \texttt{GAS} cells.
Streaming works normally for all \texttt{FLUID} cells amd for
\texttt{INTERFACE} cells we are checking whether a neighbor is GAS. If
yes, we are using outflow boundary formula in this direction.

\paragraph{Flag update}\label{flag-update}

From the mass fraction for each cell we determine whether that cell
emptied or filled. We track these cells using two arrays,
\texttt{emptiedCells} and \texttt{filledCells}. We perform the filling
and emptying in separate phases, \texttt{performFill()} and
\texttt{performEmpty()}. These phases are asymmetric in order to handle
the case of a cell filling adjacent to a cell emptying. If this happens,
the `filling' operation cancels the neighbor's `emptying' operation and
the neighboring cell is stricken from \texttt{emptiedCells}.

Both phases maintain the loop invariant of a contiguous interface layer.
When an \texttt{INTERFACE} cell is converted to \texttt{FLUID}, all
\texttt{GAS} neighbors are converted to \texttt{INTERFACE}.
Correspondingly, when an \texttt{INTERFACE} is converted to
\texttt{GAS}, all \texttt{FLUID} cells are converted to
\texttt{INTERFACE}.

\paragraph{Parallelization with
OpenMP}\label{parallelization-with-openmp}

We have added \texttt{pragma\ omp\ parallel\ for} to streaming,
collision, treatment boundaries and update flags.

\paragraph{Other optimizations}\label{other-optimizations}

At the beginning our main bottleneck was \texttt{computeFeq} function
which took approximately 42\% percent of the whole time. We splitted the
Q-loop in this function in a way that compiler was able to vectorize it,
then we got rid of all divisions and replaced them by corresponding
multiplications. Then we changed types of all double fields to float. It
gave us aproximately 2x speedup.
\end{document}


\section{Free Surfaces with LBM
D3Q19}\label{free-surfaces-with-lbm-d3q19}

Project for Computational Fluid Dynamics Lab course.

Group 6: Nathan Brei, Irving Cabrera, Natalia Saiapova.

\subsection{Compilation.}\label{compilation.}

Requirements: UNIX based OS.

Use \texttt{make} command to compile:

\begin{verbatim}
>> make
\end{verbatim}

There is debug mode available. This mode includes additional checks for
velocity and density, computes total mass in the domain and checks that
there is no FLUID cell that is neighboring with a GAS cell directly.
Also it calculates a time spent on the following parts: streaming,
collision, flag updating and boundary treatment.

\begin{verbatim}
>> make debug
\end{verbatim}

\subsection{Usage}\label{usage}

To run the program use

\begin{verbatim}
>> ./lbsim <exampleName> <number of threads>
\end{verbatim}

For each example it is required to have
\texttt{\textless{}exampleName\textgreater{}.dat} and
\texttt{\textless{}exampleName\textgreater{}.pgm} files. They have to be
placed in the \texttt{examples} directory.\\
\texttt{\textless{}number\ of\ threads\textgreater{}} specifies how many
threads will be used by OpenMP.

Output is vtk-files and they can be found in the \texttt{vtk-output}
directory (if thare is no such directory it will be created).

\subsubsection{Description of input
files}\label{description-of-input-files}

We added a few new fields to the \texttt{input.dat} file:

\begin{longtable}[]{@{}ll@{}}
\toprule
\begin{minipage}[b]{0.05\columnwidth}\raggedright\strut
Parameter\strut
\end{minipage} & \begin{minipage}[b]{0.05\columnwidth}\raggedright\strut
Description\strut
\end{minipage}\tabularnewline
\midrule
\endhead
\begin{minipage}[t]{0.05\columnwidth}\raggedright\strut
radius\strut
\end{minipage} & \begin{minipage}[t]{0.05\columnwidth}\raggedright\strut
Radius of a droplet\strut
\end{minipage}\tabularnewline
\begin{minipage}[t]{0.05\columnwidth}\raggedright\strut
exchange\_factor\strut
\end{minipage} & \begin{minipage}[t]{0.05\columnwidth}\raggedright\strut
Factor to increase speed of a mass exchnge (mainly to get more beautiful
videos)\strut
\end{minipage}\tabularnewline
\begin{minipage}[t]{0.05\columnwidth}\raggedright\strut
forces\_x\strut
\end{minipage} & \begin{minipage}[t]{0.05\columnwidth}\raggedright\strut
Gravity force in the \texttt{x} direction\strut
\end{minipage}\tabularnewline
\begin{minipage}[t]{0.05\columnwidth}\raggedright\strut
forces\_y\strut
\end{minipage} & \begin{minipage}[t]{0.05\columnwidth}\raggedright\strut
Gravity force in the \texttt{y} direction\strut
\end{minipage}\tabularnewline
\begin{minipage}[t]{0.05\columnwidth}\raggedright\strut
forces\_z\strut
\end{minipage} & \begin{minipage}[t]{0.05\columnwidth}\raggedright\strut
Gravity force in the \texttt{z} direction\strut
\end{minipage}\tabularnewline
\bottomrule
\end{longtable}

\subsection{What we have done}\label{what-we-have-done}

\paragraph{New cell types}\label{new-cell-types}

First of all we added 2 new cell types: \texttt{FLUID} and
\texttt{INTERFACE}.\\
Here is table representing cell types, which should be used in
\texttt{\textless{}exampleName\textgreater{}.pgm} file:

\begin{longtable}[]{@{}llll@{}}
\toprule
Cell type & Value & Cell type & Value\tabularnewline
\midrule
\endhead
FLUID & 0 & PRESSURE\_IN & 5\tabularnewline
INTERFACE & 1 & OBSTACLE & 6\tabularnewline
MOVING\_WALL & 2 & FREESLIP & 7\tabularnewline
INFLOW & 3 & NOSLIP & 8\tabularnewline
OUTFLOW & 4 & GAS & 9\tabularnewline
\bottomrule
\end{longtable}

\paragraph{Test functions}\label{test-functions}

In the \texttt{checks.c} one can find three check functions:

\begin{enumerate}
\tightlist
\item
  \texttt{check\_in\_rank} checks that every \texttt{FLUID} cell has a
  density between \texttt{0.9} and \texttt{1.1} and norm of the velocity
  vector is less then \texttt{sqrt(3)\ *\ C\_S};
\item
  \texttt{check\_flags} checks that no \texttt{FLUID} cell has a
  \texttt{GAS} neighbor;
\item
  \texttt{check\_mass} compute total mass in the domain (\texttt{FLUID}
  and \texttt{INTERFACE} cells).
\end{enumerate}

This checks will be activated in debug mode and be executed every time
when we write output file.

\paragraph{External forces influence}\label{external-forces-influence}

In collision step additional term for external forces was added.\\
Related function in \texttt{collision.c}:
\texttt{computeExternal(int\ i,\ float\ density,\ float\ *\ extForces)},
where \texttt{extForces} is a force vector specified in input file.

\paragraph{Mass and fluid fraction
fields}\label{mass-and-fluid-fraction-fields}

For every cell we store 2 additional float values: mass and fluid
fraction. We update mass field during streaming step. Fluid fraction
field is updated during collide step.\\
To be able get more fast we introduced \texttt{exchange\_factor} in an
input file. By this factor we increase mass exchange between cells.

\paragraph{DF from gas cells}\label{df-from-gas-cells}

In the streaming step was added reconstruction from \texttt{GAS} cells.
Streaming works normally for all \texttt{FLUID} cells amd for
\texttt{INTERFACE} cells we are checking whether a neighbor is GAS. If
yes, we are using outflow boundary formula in this direction.

\paragraph{Flag update}\label{flag-update}

From the mass fraction for each cell we determine whether that cell
emptied or filled. We track these cells using two arrays,
\texttt{emptiedCells} and \texttt{filledCells}. We perform the filling
and emptying in separate phases, \texttt{performFill()} and
\texttt{performEmpty()}. These phases are asymmetric in order to handle
the case of a cell filling adjacent to a cell emptying. If this happens,
the `filling' operation cancels the neighbor's `emptying' operation and
the neighboring cell is stricken from \texttt{emptiedCells}.

Both phases maintain the loop invariant of a contiguous interface layer.
When an \texttt{INTERFACE} cell is converted to \texttt{FLUID}, all
\texttt{GAS} neighbors are converted to \texttt{INTERFACE}.
Correspondingly, when an \texttt{INTERFACE} is converted to
\texttt{GAS}, all \texttt{FLUID} cells are converted to
\texttt{INTERFACE}.

\paragraph{Parallelization with
OpenMP}\label{parallelization-with-openmp}

We have added \texttt{pragma\ omp\ parallel\ for} to streaming,
collision, treatment boundaries and update flags.

\paragraph{Other optimizations}\label{other-optimizations}

At the beginning our main bottleneck was \texttt{computeFeq} function
which took approximately 42\% percent of the whole time. We splitted the
Q-loop in this function in a way that compiler was able to vectorize it,
then we got rid of all divisions and replaced them by corresponding
multiplications. Then we changed types of all double fields to float. It
gave us aproximately 2x speedup.
\section{Free Surfaces with LBM
D3Q19}\label{free-surfaces-with-lbm-d3q19}

Project for Computational Fluid Dynamics Lab course.

Group 6: Nathan Brei, Irving Cabrera, Natalia Saiapova.

\subsection{Compilation.}\label{compilation.}

Requirements: UNIX based OS.

Use \texttt{make} command to compile:

\begin{verbatim}
>> make
\end{verbatim}

There is debug mode available. This mode includes additional checks for
velocity and density, computes total mass in the domain and checks that
there is no FLUID cell that is neighboring with a GAS cell directly.
Also it calculates a time spent on the following parts: streaming,
collision, flag updating and boundary treatment.

\begin{verbatim}
>> make debug
\end{verbatim}

\subsection{Usage}\label{usage}

To run the program use

\begin{verbatim}
>> ./lbsim <exampleName> <number of threads>
\end{verbatim}

For each example it is required to have
\texttt{\textless{}exampleName\textgreater{}.dat} and
\texttt{\textless{}exampleName\textgreater{}.pgm} files. They have to be
placed in the \texttt{examples} directory.\\
\texttt{\textless{}number\ of\ threads\textgreater{}} specifies how many
threads will be used by OpenMP.

Output is vtk-files and they can be found in the \texttt{vtk-output}
directory (if thare is no such directory it will be created).

\subsubsection{Description of input
files}\label{description-of-input-files}

We added a few new fields to the \texttt{input.dat} file:

\begin{longtable}[]{@{}ll@{}}
\toprule
\begin{minipage}[b]{0.05\columnwidth}\raggedright\strut
Parameter\strut
\end{minipage} & \begin{minipage}[b]{0.05\columnwidth}\raggedright\strut
Description\strut
\end{minipage}\tabularnewline
\midrule
\endhead
\begin{minipage}[t]{0.05\columnwidth}\raggedright\strut
radius\strut
\end{minipage} & \begin{minipage}[t]{0.05\columnwidth}\raggedright\strut
Radius of a droplet\strut
\end{minipage}\tabularnewline
\begin{minipage}[t]{0.05\columnwidth}\raggedright\strut
exchange\_factor\strut
\end{minipage} & \begin{minipage}[t]{0.05\columnwidth}\raggedright\strut
Factor to increase speed of a mass exchnge (mainly to get more beautiful
videos)\strut
\end{minipage}\tabularnewline
\begin{minipage}[t]{0.05\columnwidth}\raggedright\strut
forces\_x\strut
\end{minipage} & \begin{minipage}[t]{0.05\columnwidth}\raggedright\strut
Gravity force in the \texttt{x} direction\strut
\end{minipage}\tabularnewline
\begin{minipage}[t]{0.05\columnwidth}\raggedright\strut
forces\_y\strut
\end{minipage} & \begin{minipage}[t]{0.05\columnwidth}\raggedright\strut
Gravity force in the \texttt{y} direction\strut
\end{minipage}\tabularnewline
\begin{minipage}[t]{0.05\columnwidth}\raggedright\strut
forces\_z\strut
\end{minipage} & \begin{minipage}[t]{0.05\columnwidth}\raggedright\strut
Gravity force in the \texttt{z} direction\strut
\end{minipage}\tabularnewline
\bottomrule
\end{longtable}

\subsection{What we have done}\label{what-we-have-done}

\paragraph{New cell types}\label{new-cell-types}

First of all we added 2 new cell types: \texttt{FLUID} and
\texttt{INTERFACE}.\\
Here is table representing cell types, which should be used in
\texttt{\textless{}exampleName\textgreater{}.pgm} file:

\begin{longtable}[]{@{}llll@{}}
\toprule
Cell type & Value & Cell type & Value\tabularnewline
\midrule
\endhead
FLUID & 0 & PRESSURE\_IN & 5\tabularnewline
INTERFACE & 1 & OBSTACLE & 6\tabularnewline
MOVING\_WALL & 2 & FREESLIP & 7\tabularnewline
INFLOW & 3 & NOSLIP & 8\tabularnewline
OUTFLOW & 4 & GAS & 9\tabularnewline
\bottomrule
\end{longtable}

\paragraph{Test functions}\label{test-functions}

In the \texttt{checks.c} one can find three check functions:

\begin{enumerate}
\tightlist
\item
  \texttt{check\_in\_rank} checks that every \texttt{FLUID} cell has a
  density between \texttt{0.9} and \texttt{1.1} and norm of the velocity
  vector is less then \texttt{sqrt(3)\ *\ C\_S};
\item
  \texttt{check\_flags} checks that no \texttt{FLUID} cell has a
  \texttt{GAS} neighbor;
\item
  \texttt{check\_mass} compute total mass in the domain (\texttt{FLUID}
  and \texttt{INTERFACE} cells).
\end{enumerate}

This checks will be activated in debug mode and be executed every time
when we write output file.

\paragraph{External forces influence}\label{external-forces-influence}

In collision step additional term for external forces was added.\\
Related function in \texttt{collision.c}:
\texttt{computeExternal(int\ i,\ float\ density,\ float\ *\ extForces)},
where \texttt{extForces} is a force vector specified in input file.

\paragraph{Mass and fluid fraction
fields}\label{mass-and-fluid-fraction-fields}

For every cell we store 2 additional float values: mass and fluid
fraction. We update mass field during streaming step. Fluid fraction
field is updated during collide step.\\
To be able get more fast we introduced \texttt{exchange\_factor} in an
input file. By this factor we increase mass exchange between cells.

\paragraph{DF from gas cells}\label{df-from-gas-cells}

In the streaming step was added reconstruction from \texttt{GAS} cells.
Streaming works normally for all \texttt{FLUID} cells amd for
\texttt{INTERFACE} cells we are checking whether a neighbor is GAS. If
yes, we are using outflow boundary formula in this direction.

\paragraph{Flag update}\label{flag-update}

From the mass fraction for each cell we determine whether that cell
emptied or filled. We track these cells using two arrays,
\texttt{emptiedCells} and \texttt{filledCells}. We perform the filling
and emptying in separate phases, \texttt{performFill()} and
\texttt{performEmpty()}. These phases are asymmetric in order to handle
the case of a cell filling adjacent to a cell emptying. If this happens,
the `filling' operation cancels the neighbor's `emptying' operation and
the neighboring cell is stricken from \texttt{emptiedCells}.

Both phases maintain the loop invariant of a contiguous interface layer.
When an \texttt{INTERFACE} cell is converted to \texttt{FLUID}, all
\texttt{GAS} neighbors are converted to \texttt{INTERFACE}.
Correspondingly, when an \texttt{INTERFACE} is converted to
\texttt{GAS}, all \texttt{FLUID} cells are converted to
\texttt{INTERFACE}.

\paragraph{Parallelization with
OpenMP}\label{parallelization-with-openmp}

We have added \texttt{pragma\ omp\ parallel\ for} to streaming,
collision, treatment boundaries and update flags.

\paragraph{Other optimizations}\label{other-optimizations}

At the beginning our main bottleneck was \texttt{computeFeq} function
which took approximately 42\% percent of the whole time. We splitted the
Q-loop in this function in a way that compiler was able to vectorize it,
then we got rid of all divisions and replaced them by corresponding
multiplications. Then we changed types of all double fields to float. It
gave us aproximately 2x speedup.
